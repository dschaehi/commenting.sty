%%%%%%%%%%%%%%%%%%%%%%%%%%%%%%%%%%%%%%%%%%%%%%%%%%%%%%%%%%%%%%%%%%%%%%%%%%%%%%%%
%
%  example.tex — Feature Walkthrough for commenting.sty
%
%  This document demonstrates all features of the commenting.sty package.
%  Compile with: pdflatex example.tex
%
%  Try changing [draft] to [final] to see how annotations disappear cleanly.
%  Try adding bg=sepia option to see the sepia background: [draft,bg=sepia]
%
%%%%%%%%%%%%%%%%%%%%%%%%%%%%%%%%%%%%%%%%%%%%%%%%%%%%%%%%%%%%%%%%%%%%%%%%%%%%%%%%

\documentclass[11pt,draft]{article}

%% Page setup
\usepackage[margin=1in]{geometry}
\usepackage{parskip}          % Paragraph spacing instead of indentation
\usepackage{booktabs}         % Nice tables
\usepackage{lipsum}           % Dummy text for demonstrations

%% Load commenting package
%% Options: draft (show comments), final (hide comments), bg=sepia (background)
\usepackage[draft]{commenting}

%%%%%%%%%%%%%%%%%%%%%%%%%%%%%%%%%%%%%%%%%%%%%%%%%%%%%%%%%%%%%%%%%%%%%%%%%%%%%%%%
%  AUTHOR REGISTRATION
%%%%%%%%%%%%%%%%%%%%%%%%%%%%%%%%%%%%%%%%%%%%%%%%%%%%%%%%%%%%%%%%%%%%%%%%%%%%%%%%
%
%  Syntax: \registerauthor{id}[Display Name][color]
%
%  Available colors (colorblind-safe):
%    cbBlue, cbOrange, cbTeal, cbRed, cbPurple,
%    cbGreen, cbYellow, cbPink, cbBrown, cbGray
%
%  This creates commands: \id, \iddel, \idi, \idf
%  And environments: idenv, idienv, idfenv

\registerauthor{alice}[Alice][cbBlue]
\registerauthor{bob}[Bob][cbOrange]
\registerauthor{carol}[Carol][cbGreen]
\registerauthor{dave}[Dave][cbPurple]

%% Note: 'todo' is pre-registered with [TODO][cbRed]

%%%%%%%%%%%%%%%%%%%%%%%%%%%%%%%%%%%%%%%%%%%%%%%%%%%%%%%%%%%%%%%%%%%%%%%%%%%%%%%%

\title{\texttt{commenting.sty} v2.2\\[0.5ex]
       \large Feature Walkthrough}
\author{Collaborative LaTeX Annotation Package}
\date{November 2025}

\begin{document}
\maketitle

\begin{abstract}
This document demonstrates all features of the \texttt{commenting.sty} package
for collaborative academic writing. Each section showcases a different feature
with explanations and examples. Change \texttt{[draft]} to \texttt{[final]} in
the package options to see how all annotations vanish cleanly.
\end{abstract}

\tableofcontents
\newpage

%%%%%%%%%%%%%%%%%%%%%%%%%%%%%%%%%%%%%%%%%%%%%%%%%%%%%%%%%%%%%%%%%%%%%%%%%%%%%%%%
\section{Quick Start}
%%%%%%%%%%%%%%%%%%%%%%%%%%%%%%%%%%%%%%%%%%%%%%%%%%%%%%%%%%%%%%%%%%%%%%%%%%%%%%%%

After loading the package and registering authors, you can immediately start
annotating:

\begin{verbatim}
\usepackage[draft]{commenting}
\registerauthor{alice}[Alice][cbBlue]

Some text with \alice{an insertion} here.
\end{verbatim}

\noindent Result: Some text with \alice{an insertion} here.

%%%%%%%%%%%%%%%%%%%%%%%%%%%%%%%%%%%%%%%%%%%%%%%%%%%%%%%%%%%%%%%%%%%%%%%%%%%%%%%%
\section{Text Additions and Deletions}
%%%%%%%%%%%%%%%%%%%%%%%%%%%%%%%%%%%%%%%%%%%%%%%%%%%%%%%%%%%%%%%%%%%%%%%%%%%%%%%%

\subsection{Added Text: \texttt{\textbackslash author\{text\}}}

Mark new text contributions with the author command. The text appears in the
author's color with a superscript signature:

\begin{itemize}
  \item \alice{Alice added this text.}
  \item \bob{Bob contributed this part.}
  \item \carol{Carol wrote this section.}
  \item \dave{Dave inserted this content.}
\end{itemize}

In \textbf{final mode}, the text remains but colors and signatures disappear.

\subsection{Deleted Text: \texttt{\textbackslash authordel\{text\}}}

Mark text for deletion with strikethrough formatting:

\begin{itemize}
  \item Original sentence \alicedel{with removed words} continues here.
  \item The \bobdel{unnecessary} method was simplified.
  \item We \caroldel{completely} revised the approach.
\end{itemize}

In \textbf{final mode}, deleted text vanishes entirely.

\subsection{Combining Additions and Deletions}

Track revisions by combining both commands:

The \bobdel{old algorithm}\bob{new method} performs \alicedel{poorly}\alice{efficiently} on large datasets.

%%%%%%%%%%%%%%%%%%%%%%%%%%%%%%%%%%%%%%%%%%%%%%%%%%%%%%%%%%%%%%%%%%%%%%%%%%%%%%%%
\section{Inline Comments}
%%%%%%%%%%%%%%%%%%%%%%%%%%%%%%%%%%%%%%%%%%%%%%%%%%%%%%%%%%%%%%%%%%%%%%%%%%%%%%%%

\subsection{Short Inline Comments: \texttt{\textbackslash authori\{comment\}}}

Inline comments appear directly in the text with highlighted backgrounds:

This paragraph needs work.\alicei{Consider restructuring this section.}
The methodology is sound,\bobi{Add more details here.} but the results
section\caroli{Missing statistical significance tests.} requires expansion.

Inline comments are ideal for:
\begin{itemize}
  \item Quick suggestions
  \item Marking areas needing attention
  \item Short questions or clarifications
\end{itemize}

In \textbf{final mode}, all inline comments disappear completely.

%%%%%%%%%%%%%%%%%%%%%%%%%%%%%%%%%%%%%%%%%%%%%%%%%%%%%%%%%%%%%%%%%%%%%%%%%%%%%%%%
\section{Footnote Comments}
%%%%%%%%%%%%%%%%%%%%%%%%%%%%%%%%%%%%%%%%%%%%%%%%%%%%%%%%%%%%%%%%%%%%%%%%%%%%%%%%

\subsection{Short Footnote Comments: \texttt{\textbackslash authorf\{comment\}}}

For longer comments that would disrupt reading flow, use footnotes:

The experiment was conducted over six months\alicef{We should mention the
specific dates and any interruptions that occurred during this period.} with
careful monitoring of all variables\bobf{Consider adding a table summarizing
the monitored variables and their acceptable ranges.}.

Footnote comments feature:
\begin{itemize}
  \item Alphabetic markers (a, b, c, ...) to distinguish from regular footnotes
  \item Clickable backlinks (click the marker in the footnote to return)
  \item Author-colored text for easy identification
\end{itemize}

\subsection{Comparison: Inline vs. Footnote}

\begin{center}
\begin{tabular}{lcc}
\toprule
\textbf{Feature} & \textbf{Inline} (\texttt{\textbackslash Xi}) & \textbf{Footnote} (\texttt{\textbackslash Xf}) \\
\midrule
Visibility & In-text & Page bottom \\
Best for & Short notes & Long explanations \\
Reading flow & Interrupts slightly & Preserves flow \\
\bottomrule
\end{tabular}
\end{center}

%%%%%%%%%%%%%%%%%%%%%%%%%%%%%%%%%%%%%%%%%%%%%%%%%%%%%%%%%%%%%%%%%%%%%%%%%%%%%%%%
\section{Multi-line Comment Environments}
%%%%%%%%%%%%%%%%%%%%%%%%%%%%%%%%%%%%%%%%%%%%%%%%%%%%%%%%%%%%%%%%%%%%%%%%%%%%%%%%

\subsection{Multi-line Inline: \texttt{authorienv}}

For extended inline discussions, use the \texttt{ienv} environment:

\begin{aliceienv}
This is a multi-line inline comment that spans several lines.
It's perfect for detailed suggestions, extended discussions,
or when you need to explain a complex revision proposal.

The comment appears in a colored box with the author's signature
in the margin, making it easy to identify at a glance.
\end{aliceienv}

Here is how Bob might respond:

\begin{bobienv}
I agree with Alice's suggestion above. We should also consider
the implications for Section 5, which relies heavily on the
assumptions made here.
\end{bobienv}

\subsection{Multi-line Footnote: \texttt{authorfenv}}

For lengthy footnote discussions that don't fit in a single line:

The theoretical framework builds on previous work%
\begin{alicefenv}
This entire paragraph needs citations. Key references include:
Smith (2020) for the foundational theory,
Jones (2021) for the methodological approach,
and Brown (2022) for the statistical validation.
\end{alicefenv}
and extends it to new domains.

%%%%%%%%%%%%%%%%%%%%%%%%%%%%%%%%%%%%%%%%%%%%%%%%%%%%%%%%%%%%%%%%%%%%%%%%%%%%%%%%
\section{Colored Block Environments}
%%%%%%%%%%%%%%%%%%%%%%%%%%%%%%%%%%%%%%%%%%%%%%%%%%%%%%%%%%%%%%%%%%%%%%%%%%%%%%%%

\subsection{Author Blocks: \texttt{authorenv}}

Mark entire paragraphs as contributions from a specific author:

\begin{aliceenv}
This paragraph was written entirely by Alice. The colored text and margin
signature clearly identify the author. This is useful for tracking who wrote
which sections during collaborative editing, or for marking sections that are
under review.
\end{aliceenv}

\begin{bobenv}
Bob's contribution follows here. Notice how each author's block uses their
assigned color from the colorblind-safe palette. This ensures that all
collaborators, including those with color vision deficiency, can distinguish
between different authors' contributions.
\end{bobenv}

\begin{carolenv}
Carol adds her section here. The margin signatures make it easy to scan the
document and quickly identify who wrote what, even when printing in grayscale.
\end{carolenv}

%%%%%%%%%%%%%%%%%%%%%%%%%%%%%%%%%%%%%%%%%%%%%%%%%%%%%%%%%%%%%%%%%%%%%%%%%%%%%%%%
\section{TODO Lists}
%%%%%%%%%%%%%%%%%%%%%%%%%%%%%%%%%%%%%%%%%%%%%%%%%%%%%%%%%%%%%%%%%%%%%%%%%%%%%%%%

The package includes a \texttt{todolist} environment with checkbox symbols:

\begin{todolist}
  \item Write introduction
  \item[\doing] Revise methodology section
  \item[\done] Complete literature review
  \item Run additional experiments
  \item[\done] Update figures
\end{todolist}

Checkbox commands:
\begin{itemize}
  \item Default: empty checkbox ($\square$)
  \item \texttt{\textbackslash doing}: in-progress marker
  \item \texttt{\textbackslash done}: completed marker
\end{itemize}

\subsection{Pre-registered TODO Author}

The package pre-registers a \texttt{todo} author for quick task marking:

This section is incomplete.\todoi{Add experimental results here.}

The algorithm needs optimization.\todof{Benchmark against baseline methods
and report timing comparisons in Table 3.}

%%%%%%%%%%%%%%%%%%%%%%%%%%%%%%%%%%%%%%%%%%%%%%%%%%%%%%%%%%%%%%%%%%%%%%%%%%%%%%%%
\section{Colorblind-Safe Palette}
%%%%%%%%%%%%%%%%%%%%%%%%%%%%%%%%%%%%%%%%%%%%%%%%%%%%%%%%%%%%%%%%%%%%%%%%%%%%%%%%

The package provides 10 colors optimized for color vision deficiency (CVD):

\begin{center}
\begin{tabular}{ll}
\toprule
\textbf{Color Name} & \textbf{Sample} \\
\midrule
\texttt{cbBlue}   & \textcolor{cbBlue}{Sample text in blue} \\
\texttt{cbOrange} & \textcolor{cbOrange}{Sample text in orange} \\
\texttt{cbTeal}   & \textcolor{cbTeal}{Sample text in teal} \\
\texttt{cbRed}    & \textcolor{cbRed}{Sample text in red} \\
\texttt{cbPurple} & \textcolor{cbPurple}{Sample text in purple} \\
\texttt{cbGreen}  & \textcolor{cbGreen}{Sample text in green} \\
\texttt{cbYellow} & \textcolor{cbYellow}{Sample text in yellow} \\
\texttt{cbPink}   & \textcolor{cbPink}{Sample text in pink} \\
\texttt{cbBrown}  & \textcolor{cbBrown}{Sample text in brown} \\
\texttt{cbGray}   & \textcolor{cbGray}{Sample text in gray} \\
\bottomrule
\end{tabular}
\end{center}

%%%%%%%%%%%%%%%%%%%%%%%%%%%%%%%%%%%%%%%%%%%%%%%%%%%%%%%%%%%%%%%%%%%%%%%%%%%%%%%%
\section{Package Options}
%%%%%%%%%%%%%%%%%%%%%%%%%%%%%%%%%%%%%%%%%%%%%%%%%%%%%%%%%%%%%%%%%%%%%%%%%%%%%%%%

\subsection{Draft vs. Final Mode}

\begin{verbatim}
\usepackage[draft]{commenting}  % Show all annotations
\usepackage[final]{commenting}  % Hide all annotations
\end{verbatim}

The package also inherits the document class option:
\begin{verbatim}
\documentclass[draft]{article}  % commenting.sty uses draft mode
\documentclass[final]{article}  % commenting.sty uses final mode
\end{verbatim}

\subsection{Sepia Background}

For comfortable reading during long editing sessions:
\begin{verbatim}
\usepackage[draft,bg=sepia]{commenting}
\end{verbatim}

This applies a warm sepia tone to the page background in draft mode.

%%%%%%%%%%%%%%%%%%%%%%%%%%%%%%%%%%%%%%%%%%%%%%%%%%%%%%%%%%%%%%%%%%%%%%%%%%%%%%%%
\section{Complete Example: Collaborative Revision}
%%%%%%%%%%%%%%%%%%%%%%%%%%%%%%%%%%%%%%%%%%%%%%%%%%%%%%%%%%%%%%%%%%%%%%%%%%%%%%%%

Here's a realistic example showing how multiple authors might collaborate:

\begin{aliceenv}
The proposed method achieves \alice{state-of-the-art} results on
\alicedel{all}\alice{most} benchmark datasets.\alicei{Need to verify the claim
on Dataset X.} Our implementation\alicef{Code will be released upon
publication at github.com/example/project.} demonstrates significant
improvements over \bobdel{existing}\bob{previous} approaches.
\end{aliceenv}

\begin{bobienv}
I've reviewed Alice's revisions above. The claim about ``state-of-the-art''
may be too strong---we should qualify it or provide supporting evidence
from our ablation studies.
\end{bobienv}

\begin{carolfenv}
Agreed with Bob. I'll add the ablation study results to Table 2 and update
the claims accordingly. Also, we need to address Reviewer 2's concern about
computational complexity.
\end{carolfenv}

%%%%%%%%%%%%%%%%%%%%%%%%%%%%%%%%%%%%%%%%%%%%%%%%%%%%%%%%%%%%%%%%%%%%%%%%%%%%%%%%
\section{Summary of Commands}
%%%%%%%%%%%%%%%%%%%%%%%%%%%%%%%%%%%%%%%%%%%%%%%%%%%%%%%%%%%%%%%%%%%%%%%%%%%%%%%%

For each registered author \texttt{X}:

\begin{center}
\begin{tabular}{lll}
\toprule
\textbf{Command/Environment} & \textbf{Purpose} & \textbf{Final Mode} \\
\midrule
\texttt{\textbackslash X\{text\}}           & Added text        & Text remains \\
\texttt{\textbackslash Xdel\{text\}}        & Deleted text      & Text removed \\
\texttt{\textbackslash Xi\{comment\}}       & Inline comment    & Hidden \\
\texttt{\textbackslash Xf\{comment\}}       & Footnote comment  & Hidden \\
\texttt{Xenv}                               & Colored block     & Plain text \\
\texttt{Xienv}                              & Multi-line inline & Hidden \\
\texttt{Xfenv}                              & Multi-line footnote & Hidden \\
\bottomrule
\end{tabular}
\end{center}

\vfill
\begin{center}
\rule{0.5\textwidth}{0.4pt}\\[1ex]
\textit{End of feature walkthrough}
\end{center}

\end{document}
