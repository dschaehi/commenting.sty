\documentclass{article}
\usepackage[draft]{commenting}
\usepackage{lipsum}  % For demo text

% Register reviewers with different roles
\registerauthor{alice}{cbBlue}      % Technical editor
\registerauthor{bob}{cbOrange}      % Content reviewer
\registerauthor{carol}{cbTeal}      % Copy editor
\registerauthor{dave}{cbPurple}     % Format checker

\title{Using \texttt{commenting.sty}: An Overview}
\author{Jae Hee Lee}
\date{\today}

\begin{document}
\maketitle

\section{Package Overview}
The \texttt{commenting.sty} package provides a comprehensive solution for collaborative document 
review in LaTeX\bobfootnote{Add package version number}. It supports multiple reviewers, each 
with their own color-coded comments\aliceadd{, ensuring clear attribution of feedback}.

\section{Comment Types}
The package offers four main types of review annotations:

\begin{enumerate}
    \item \textbf{Inline Comments:} Quick notes that appear directly in 
    the text\daveinline{Like this inline comment}
    
    \item \textbf{Footnote Comments:} Detailed observations that appear at the page 
    bottom\bobfootnote{Footnotes are perfect for longer explanations}
    
    \item \textbf{Text Additions:} Track suggested insertions with 
    \caroladd{highlighted text and reviewer attribution}
    
    \item \textbf{Text-Additions Environment:} The environment version of text additions
\end{enumerate}

\section{Implementation Examples}
\begin{aliceenv}
Here's how to implement comments in your document:

1. Register an author with color:
   \verb|\registerauthor{<author>}{<color>}|

2. This creates four commands automatically:
   \begin{itemize}
       \item \verb|\<author>|\verb|inline{text}| for inline comments
       \item \verb|\<author>|\verb|footnote{text}| for comments as footnotes
       \item \verb|\<author>|\verb|add{text}| for additions
       \item \verb|\begin{<author>|\verb|env}| environment
   \end{itemize}
\end{aliceenv}

\section{Text-Addition Environment}
Text additon environments are perfect for longer additions of text\bobfootnote{Each reviewer's environment uses their assigned color}.

\begin{aliceenv}
The syntax for environments is:

\begin{verbatim}
\begin{<author>env}
    Your content goes here...
    Can include multiple paragraphs
    and other LaTeX commands
\end{<author>env}
\end{verbatim}
\end{aliceenv}

\begin{bobenv}
Environments can contain:
\begin{itemize}
    \item Regular text
    \item Other LaTeX environments
    \item \caroladd{Nested comments from other reviewers}
    \item Mathematical equations
    \item Tables and figures
\end{itemize}
\end{bobenv}

\section{Color System}
The package uses a carefully selected colorblind-safe palette\davefootnote{Based on 
ColorBrewer scheme}:

\begin{itemize}
    \item \textcolor{cbBlue}{Blue} and \textcolor{cbOrange}{orange} for primary reviewers
    \item \textcolor{cbTeal}{Teal} and \textcolor{cbPurple}{purple} for secondary reviewers
    \item Additional colors available: \textcolor{cbRed}{red}, \textcolor{cbGreen}{green}, 
    \textcolor{cbPink}{pink}\aliceinline{Add complete color list in documentation}
\end{itemize}

\section{Usage Guidelines}
To use the package effectively:
\begin{enumerate}
    \item Load the package with \verb|\usepackage[draft]{commenting}|
    \item Register all reviewers at the document start
    \item \caroladd{Use consistent comment types for similar feedback}
    \item Switch to \verb|[final]| option for the publication version
\end{enumerate}

\section{Advanced Features}
The package also supports\bobinline{Expand this section} advanced features like:
\begin{itemize}
    \item Multiple footnote styles
    \item Custom color definitions
    \item \caroladd{Conditional comment visibility}
    \item Integration with version control
\end{itemize}

\end{document}
