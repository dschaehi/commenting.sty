\documentclass[10pt]{article}
\usepackage[margin=0.75in]{geometry}
\usepackage[draft]{commenting}

% Register authors with different color configurations
\registerauthor{alice}[Alice][cbBlue]
\registerauthor{bob}[Bob][cbOrange]
\registerauthor{carol}[none][cbGreen]  % No signature display

\title{\texttt{commenting.sty} v2.1 Feature Demo}
\author{Example Document}
\date{}

\begin{document}
\maketitle
\vspace{-1em}

\section{Text Tracking}

Original text with \alice{added content} and \bobdel{deleted material}. Multiple authors can \bob{contribute} to the same sentence.

\section{Comment Types}

Inline comments\alicei{like this} appear directly in text. Footnote comments\bobf{appear at bottom} keep the main text clean. The package supports\carolf{anonymous comments too} various configurations.

\section{Multi-line Comments}

\begin{aliceienv}
Multi-line inline comments span several lines,
perfect for detailed suggestions or discussions.
\end{aliceienv}

Regular text continues here\begin{bobfenv}
while multi-line footnote comments
provide extended explanations
without cluttering the main text.
\end{bobfenv}.

\section{Colored Environments}

\begin{aliceenv}
Entire paragraphs can be highlighted with author colors and margin signatures. This is useful for marking sections under review or highlighting specific contributions.
\end{aliceenv}

\begin{bobenv}
Different authors get different colors from the colorblind-safe palette, ensuring accessibility for all collaborators.
\end{bobenv}

\section{Combined Features}

In practice, authors combine \alice{additions}, \alicedel{deletions}, \alicei{quick notes}, and detailed comments\alicef{with full citations or explanations} to create a rich collaborative environment.

\begin{carolienv}
Even authors without displayed signatures
can contribute multi-line comments effectively.
\end{carolienv}

\vspace{0.5em}
\noindent\textbf{Note:} In final mode (\texttt{[final]}), all comments vanish and only the base text with additions remains.

\end{document}
